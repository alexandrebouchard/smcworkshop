Gaussian particle filters replace the resampling process of traditional
particle filters by the computation of Gaussians, which are then used
for drawing auxiliary particles that serve as parents for the particles
generated in the next time step. Gaussian particle filters have some
advantages over standard particle filters including in hardware
implementation and in treating constant parameters. In this paper,
we present a few novel implementations of Gaussian particle filters,
and show how they relate to each other. We compare their
performance in stressful settings where the dimension of the state
vector and number of static parameters of the model are high.