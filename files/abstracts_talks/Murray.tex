We introduce a dynamic mechanism for the solution of analytically-tractable substructure in probabilistic programs. For inference with
Sequential Monte Carlo, it automatically yields improvements such as
locally-optimal proposals and Rao-Blackwellization. It works by
maintaining a directed graph alongside the running program,
evolving dynamically as the program triggers operations upon it.
Nodes of the graph represent random variables, and edges the
analytically-tractable relationships between them (e.g. conjugate
priors and affine transformations). Each random variable is held in the
graph for as long as possible, and is sampled only when used by the
program in a context that cannot be resolved analytically. This allows
it to be conditioned on as many observations as possible before sampling. The approach has been implemented in a new probabilistic
programming language called Birch, in which several examples will be
given.