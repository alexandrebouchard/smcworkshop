The Ensemble Kalman filter is a sophisticated and powerful
data assimilation method for filtering high dimensional problems
arising in fluid mechanics and geophysical sciences. This Monte Carlo
method can be interpreted as a mean-field McKean-Vlasov type
particle interpretation of the Kalman-Bucy difusions. In contrast to
more conventional particle filters and nonlinear Markov processes
these models are designed in terms of a difusion process with a
difusion matrix that depends on particle covariance matrices.
Besides some recent advances on the stability of nonlinear Langevin
type difusions with drift interactions, the long-time behaviour of
models with interacting difusion matrices and conditional
distribution interaction functions has never been discussed in the
literature. One of the main contributions of the talk is to initiate the
study of this new class of models. The talk presents a series of new
functional inequalities to quantify the stability of these nonlinear difusion processes. In the same vein, despite some recent
contributions on the convergence of the Ensemble Kalman filter when
the number of sample tends to infinity very little is known on stability
and the long-time behaviour of these mean-field interacting type
particle filters. The second contribution of this talk is to provide
uniform propagation of chaos properties as well as Lp-mean error
estimates w.r.t. to the time horizon. Our regularity condition is also
shown to be sufcient and necessary for the uniform convergence of
the Ensemble Kalman filter. The stochastic analysis developed in this
talk is based on an original combination of functional inequalities and
Foster-Lyapunov techniques with coupling, martingale techniques,
random matrices and spectral analysis theory.