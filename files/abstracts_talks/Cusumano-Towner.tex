Although sequential Monte Carlo techniques are broadly applicable in principle, they can be difficult to apply in practice. Two challenges faced by practitioners are (i) correctly implementing state-of-the-art SMC strategies, and (ii) measuring the approximation error of any given SMC scheme. This talk will present research aimed at addressing these two problems, building on ideas from probabilistic programming. First, this talk will introduce Gen, a new probabilistic programming platform that makes it easier to write complex inference algorithms that include use of SMC. Gen programs can specify generative models and custom inference strategies for those models, including proposals based on variational inference, deep learning, and proposals that are themselves programs written in Gen. Gen will be illustrated using inverse planning problems drawn from probabilistic robotics. Second, this talk will introduce a new algorithm for measuring the accuracy of approximate inference algorithms, based on a new estimator for the symmetric KL divergence between a general SMC sampler and its target distribution. This estimator, called AIDE, is based on the idea that Monte Carlo inference algorithms can be viewed as probabilistic models whose internal random choices correspond to auxiliary variables. AIDE will be illustrated on synthetic hidden Markov models and Dirichlet process mixture model examples.