\documentclass[12pt]{article}

\usepackage{amsmath}
\usepackage{url}
\newcommand{\postertitle}[1]{{\Large\bf #1}\\[12pt]}
\newcommand{\authors}[1]{\emph{#1}\\}
\newcommand{\affiliations}[1]{{#1}\\}
\newcommand{\contacts}[1]{{#1}}

%%%%%%%%%%%%%%%%%%%%%%%%%%%%%%%%%%%%%%%%%%%%%%%%%%%%%%%%%%%%%%%%%%%%%%
\begin{document}

\begin{center}
\vspace*{0.5cm}
%
\postertitle{Upper and lower bounds on kernel density estimates in particle filtering}
%
\authors{David Coufal$^\star$} % please mark the name of the person(s) presenting the poster with a star
% 
\affiliations{
Charles University in Prague, Faculty of Mathematics and Physics\\
Department of Probability and Mathematical Statistics\\
Sokolovsk\'{a} 83, 186 75 Praha 8, Czech Republic
}
%
\contacts{e-mail: coufal@karlin.mff.cuni.cz} % URL or email address of contact person
%
\vspace*{0.3cm}
\end{center}

%%%%%%%%%%   Type your abstract below
The upper bounds on kernel density estimates in particle filtering are derived for the Sobolev class of filtering densities. 
The bounds assure the convergence of kernel density estimates to the filtering density at each filtering time, provided 
the number of generated particles goes to infinity. The result is achieved by means of Fourier analysis. 
The lower bounds are discussed as well. They are delivered using the standard approach from non-parameteric
estimation that employs tools from information theory. The lower bounds meet upper ones, which indicates
optimality of the kernel density estimates in particle filtering.
%%%%%%%%%%%   References
%If you have references, you can produce a .bbl file using bibtex
%and copy/paste thebibliography here
\begin{thebibliography}{1}	
	\bibitem{Coufal2016}
	D.~Coufal
	\newblock On convergence of kernel density estimates in particle filtering. 
	\newblock {\em Kybernetika}, 5(52):735--756, 2016.\\
	\newblock \texttt{http://www.kybernetika.cz/content/2016/5/735}
\end{thebibliography}


\end{document}