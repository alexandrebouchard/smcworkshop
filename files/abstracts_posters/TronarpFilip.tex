\documentclass[12pt]{article}

\usepackage{amsmath}
\usepackage{url}
\newcommand{\postertitle}[1]{{\Large\bf #1}\\[12pt]}
\newcommand{\authors}[1]{\emph{#1}\\}
\newcommand{\affiliations}[1]{{#1}\\}
\newcommand{\contacts}[1]{{#1}}

%%%%%%%%%%%%%%%%%%%%%%%%%%%%%%%%%%%%%%%%%%%%%%%%%%%%%%%%%%%%%%%%%%%%%%
\begin{document}

\begin{center}
\vspace*{0.5cm}
%
\postertitle{Rao-Blackwellized Particle Implementation Of Stochastic Expectation Maximization In MEG/EEG for Joint Estimation of Neural Sources and Connectivity}
%
%\authors{Fredrik Lindsten$^\star$ and Thomas B. Sch\"on} % please mark the name of the person(s) presenting the poster with a star
% 
\authors{Narayan Subramaniyam$^1$, Sara Sommariva$^1$, Filip Tronarp$^{2\star}$, Xi Chen$^3$, Simo S\"arkk\"a$^2$ and Lauri Parkkonen$^1$}
%
\affiliations{$^1$ Aalto University, Department of Neuroscience and Biomedical Engineering}
\affiliations{$^2$ Aalto University, Department of Eleictrical Engineering and Automation}
\affiliations{$^3$ Cavendish Laboratory, University of Cambridge}
%
\contacts{\url{filip.tronarp@aalto.fi}} % URL or email address of contact person
%
\vspace*{0.3cm}
\end{center}

%%%%%%%%%%   Type your abstract below
Functional connectivity is an important topic in neuroscience which can serve as a tool for basic research as well as a clinical tool for diagnosis by, for instance, aiding in localisation of brain regions generating epilleptic seizures. Estimating the functional connectivity involves four components (i) determining the number of electromagnetic sources (ii) determining the location of the sources, (iii) determining the strength (dipole moment) of the sources and (iv) determining their statistical interdependence. The current standard in the neuroscience community is to use a two-step approach where (ii) and (iii) are solved by, e.g, minimum norm estimation after which (iv) can be solved, however this induces a bias \cite{EMBEC17}. Here the problem is tackled by assuming the solution to (i) is known a priori while a state-space model is formulated for (ii) and (iii) given (iv), where the latter is assumed to impose linear dependence between sources. The problem is thus reduced to identification in a non-linear state-space model with a linear sub-structure which is solved by a Rao-Blackwellized particle filter combined with an expectation-maximisation algorithm. The proposed approach allows for reconstruction of neural sources and connectivity in a realistic and continuous head model \cite{XiChen}.  


%%%%%%%%%%%   References
%If you have references, you can produce a .bbl file using bibtex
%and copy/paste thebibliography here
\begin{thebibliography}{1}	
\bibitem{EMBEC17}
NP Subramaniyam and F Tronarp and S S\"arkk\"a and L Parkkonen,
\textit{Expectation–-maximization algorithm with a nonlinear Kalman smoother for MEG/EEG connectivity estimation},
EMBEC \& NBC 2017. 

\bibitem{XiChen}
Xi Chen and Simo S\"arkk\"a and Simon Godsill, 
\textit{A Bayesian particle filtering method for brain source localisation},
Digital Signal Processing, 47:192-204, 2015. 

\end{thebibliography}


\end{document}
