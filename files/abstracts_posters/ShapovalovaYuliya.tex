\documentclass[12pt]{article}

\usepackage{amsmath}
\usepackage{url}
\newcommand{\postertitle}[1]{{\Large\bf #1}\\[12pt]}
\newcommand{\authors}[1]{\emph{#1}\\}
\newcommand{\affiliations}[1]{{#1}\\}
\newcommand{\contacts}[1]{{#1}}

%%%%%%%%%%%%%%%%%%%%%%%%%%%%%%%%%%%%%%%%%%%%%%%%%%%%%%%%%%%%%%%%%%%%%%
\begin{document}

\begin{center}
\vspace*{0.5cm}
%
\postertitle{Volatility Spillovers with Multivariate Stochastic Volatility Models}
%
\authors{Yuliya Shapovalova$^\star$ and Michael Eichler} % please mark the name of the person(s) presenting the poster with a star
% 
\affiliations{Department of Quantitative Economics, Maastricht University}
%
\contacts{\url{y.shapovalova@maastrichtuniversity.nl}} % URL or email address of contact person
%
\vspace*{0.3cm}
\end{center}

%%%%%%%%%%   Type your abstract below
Co-movements in financial time series suggest the presence of volatility spillover effects among financial
markets. Understanding fundamentals behind this phenomena is important for portfolio
managers and policy makers. Currently in the literature GARCH-type models is the dominating
approach for detecting volatility spillovers. The inference is often based on notions of causality
in mean and variance. In this paper, we aim to analyze volatility spillovers using a more natural
approach for volatility modeling: multivariate stochastic volatility models (MSVM). The
structure of MSVM allows to test for causality in volatility processes directly, and in contrast to
GARCH models, causality in variance and causality in volatility do not coincide in this framework.
However, due to the presence of latent volatility processes estimation of this class of
models is a difficult task. We start with adopting an off-shelf solution – bootstrap and auxiliary
particle filters in Particle Markov Chain Monte Carlo setting, and discuss limitations that arise
in the multivariate case. We further discuss possibilities to improve the off-shelf methodology,
by using different particle filtering schemes (particle efficient importance sampling or iterated
auxiliary particle filter), adopting Hamiltonian Monte Carlo methods and combining pMCMC 
with Bayesian approximate methods to
improve computational efficiency and speed up the convergence.


%%%%%%%%%%%   References
%If you have references, you can produce a .bbl file using bibtex
%and copy/paste thebibliography here
%\begin{thebibliography}{1}	
%	\bibitem{label}
%	A.~Anderson 
%	\newblock Novel theory for methods
%	\newblock {\em Journal of Theory and Methods}, 1(2):1--12, 2017.	
%\end{thebibliography}


\end{document}
