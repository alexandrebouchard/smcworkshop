\documentclass[12pt]{article}

\usepackage{amsmath}
\usepackage{url}
\newcommand{\postertitle}[1]{{\Large\bf #1}\\[12pt]}
\newcommand{\authors}[1]{\emph{#1}\\}
\newcommand{\affiliations}[1]{{#1}\\}
\newcommand{\contacts}[1]{{#1}}

%%%%%%%%%%%%%%%%%%%%%%%%%%%%%%%%%%%%%%%%%%%%%%%%%%%%%%%%%%%%%%%%%%%%%%
\begin{document}

\begin{center}
\vspace*{0.5cm}
%
\postertitle{A probabilistic scheme for joint parameter estimation and state prediction in complex dynamical systems}
%
\authors{Sara P\'erez-Vieites$^\star$, In\'es P. Mari\~no and Joaqu\'{\i}n M\'{\i}guez} % please mark the name of the person(s) presenting the poster with a star
% 
\affiliations{Department of Signal Theory \& Communications, Universidad Carlos III de Madrid}
%
\contacts{\url{spvieites@tsc.uc3m.es}} % URL or email address of contact person
%
\vspace*{0.3cm}
\end{center}

%%%%%%%%%%   Type your abstract below
Many problems in the geophysical sciences demand the ability to calibrate the parameters and predict the time evolution of complex dynamical models using sequentially-collected data. Here we introduce a general methodology for the joint estimation of the static parameters and the forecasting of the state variables of nonlinear, and possibly chaotic, dynamical models. The proposed scheme is essentially probabilistic. It aims at recursively computing the sequence of joint posterior probability distributions of the unknown model parameters and its (time varying) state variables conditional on the available observations. The latter are possibly partial and contaminated by noise. The new framework combines a Monte Carlo scheme to approximate the posterior distribution of the fixed parameters with filtering (or data assimilation) techniques to track and predict the distribution of the state variables. For this reason, we refer to the proposed methodology as nested filtering. In this paper we specifically explore the use of Gaussian filtering methods, but other approaches fit naturally within the new framework. As an illustrative example, we apply three different implementations of the methodology to the tracking of the state, and the estimation of the fixed parameters, of a stochastic two-scale Lorenz 96 system. This model is commonly used to assess data assimilation procedures in meteorology. For this example, we compare different nested filters and show estimation and forecasting results for a 4,000-dimensional system.

%The abstract goes here. Maybe with a reference \cite{label} and some math:
%\begin{align*}
%	\exp(i\pi) = -1
%\end{align*}
%
%
%%%%%%%%%%%%   References
%%If you have references, you can produce a .bbl file using bibtex
%%and copy/paste thebibliography here
%\begin{thebibliography}{1}	
%	\bibitem{label}
%	A.~Anderson 
%	\newblock Novel theory for methods
%	\newblock {\em Journal of Theory and Methods}, 1(2):1--12, 2017.	
%\end{thebibliography}


\end{document}
