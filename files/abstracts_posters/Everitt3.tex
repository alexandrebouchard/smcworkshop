\documentclass[12pt]{article}

\usepackage{amsmath}
\usepackage{url}
\newcommand{\postertitle}[1]{{\Large\bf #1}\\[12pt]}
\newcommand{\authors}[1]{\emph{#1}\\}
\newcommand{\affiliations}[1]{{#1}\\}
\newcommand{\contacts}[1]{{#1}}

%%%%%%%%%%%%%%%%%%%%%%%%%%%%%%%%%%%%%%%%%%%%%%%%%%%%%%%%%%%%%%%%%%%%%%
\begin{document}

\begin{center}
\vspace*{0.5cm}
%
\postertitle{A rare event approach to high dimensional Approximate Bayesian computation}
%
\authors{Richard Everitt} % please mark the name of the person(s) presenting the poster with a star
%
\vspace*{0.3cm}
\end{center}

Approximate Bayesian computation (ABC) methods permit approximate inference for intractable likelihoods when it is possible to simulate from the model. However they perform poorly for high dimensional data, and in practice must usually be used in conjunction with dimension reduction methods, resulting in a loss of accuracy which is hard to quantify or control. We propose a new ABC method for high dimensional data based on rare event methods which we refer to as RE-ABC. This uses a latent variable representation of the model. For a given parameter value, we estimate the probability of the rare event that the latent variables correspond to data roughly consistent with the observations. This is performed using sequential Monte Carlo and slice sampling to systematically search the space of latent variables. In contrast standard ABC can be viewed as using a more naive Monte Carlo estimate. We use our rare event probability estimator as a likelihood estimate within the pseudo-marginal Metropolis-Hastings algorithm for parameter inference. 
We provide asymptotics showing that RE-ABC has a lower computational cost for high dimensional data than standard ABC methods. We also illustrate our approach empirically, on a Gaussian distribution and an application in infectious disease modelling.

This is joint work with Dennis Prangle and Theodore Kypraios and is available at arxiv.org/abs/1611.02492



\end{document}
