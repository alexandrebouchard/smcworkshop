\documentclass[12pt]{article}

\usepackage{amsmath}
\usepackage{url}
\newcommand{\postertitle}[1]{{\Large\bf #1}\\[12pt]}
\newcommand{\authors}[1]{\emph{#1}\\}
\newcommand{\affiliations}[1]{{#1}\\}
\newcommand{\contacts}[1]{{#1}}

%%%%%%%%%%%%%%%%%%%%%%%%%%%%%%%%%%%%%%%%%%%%%%%%%%%%%%%%%%%%%%%%%%%%%%
\begin{document}

\begin{center}
\vspace*{0.5cm}
%
\postertitle{Multi-dipole estimation from (simultaneous) MEG and EEG data: a semi-analytic
Sequential Monte Carlo approach.}
%
\authors{Sara Sommariva$^{\star, 1}$ and Gianvittorio Luria$^2$ and Alberto Sorrentino$^2$} % please mark the name of the person(s) presenting the poster with a star
% 
\affiliations{1.Department of Neuroscience and Biomedical Engineering, Aalto University \\
2.Department of Mathematics, Universit\`a di Genova}
%
\contacts{\url{sara.sommariva@aalto.fi}} % URL or email address of contact person
%
\vspace*{0.3cm}
\end{center}

%%%%%%%%%%   Type your abstract below
From a macroscopic point of view, any brain function is characterized by the activation and
interaction of specific brain areas.\\
Magnetoencephalography (MEG) and Electroencephalography (EEG) are two modern
neuroimaging techniques featuring an outstanding temporal resolution which makes them
particularly suitable to study such functional networks. To this aim, MEG/EEG data are
typically analyzed in two steps. First, the neural currents that have generated the recorded
data are estimated (\textit{source reconstruction}); then, the statistical relationship between 
the activity of the involved brain areas is computed (\textit{functional connectivity analysis}).\\

\noindent
In \cite{soso14} and \cite{soetal14} we have shown how the Sequential Monte Carlo (SMC) approach introduced in
\cite{DelMoetal06} can be used to perform the first step of the process, namely source reconstruction.
We modelled neural sources as the superimposition of an unknown number of point-like
sources, termed dipoles, each one representing the activity of a small brain area. Both the
number and the locations of the dipoles are assumed to remain fixed in time, while their
moments, i.e. their intensities and orientations, are allowed to change.
Under these hypothesis both the MEG and EEG forward problems can be written in terms of
a conditionally linear model. Indeed, data depend linearly on the dipole moments, while
depend non-linearly on their number and locations. Assuming a Gaussian prior for the linear
variables and for the noise probability distribution, we used the SMC approach to
approximate the posterior distribution of the number of sources and their locations while we
computed analytically the conditional posterior of the time-varying dipole moments.
Moreover, in \cite{roetal17} we have shown how, setting properly the noise covariance matrix, the
presented algorithm can be used for the joint analysis of MEG and EEG data, resulting in a
lower variance of the posterior distribution and thus in a lower uncertainty on the solution.
Our next goal is to study the application of this algorithm for connectivity analysis, i.e. for
recovering the dynamic interaction between the different active dipoles together with their
time-courses.


%%%%%%%%%%%   References
%If you have references, you can produce a .bbl file using bibtex
%and copy/paste thebibliography here
\begin{thebibliography}{1}	
	\bibitem{soso14}
	S.~Sommariva and A.~Sorrentino 
	\newblock Sequential Monte Carlo samplers for semi-linear inverse problems and application to Magnetoencephalography 
	\newblock {\em Inverse Problems}, 30 114020, 2014.
	\bibitem{soetal14}
	A.~Sorrentino, G.~Luria and R.~Aramini  
	\newblock Bayesian Multi-Dipole Modeling of a Single Topography in MEG by Adaptive Sequential Monte Carlo Samplers 
	\newblock {\em Inverse Problems}, 30 045010, 2014
	\bibitem{DelMoetal06}
	P.~Del Moral, A.~Doucet and A.~Jasra  
	\newblock Sequential Monte Carlo samplers 
	\newblock {\em Journal Of The Royal Statistical Society Series B}, 68 41--36, 2006
	\bibitem{roetal17}
	F.~Rossi, G.~Luria, S.~Sommariva and A.~Sorrentino   
	\newblock Bayesian multi-dipole localization and uncertainty quantification from simultaneous EEG and MEG recordings 
	\newblock {\em EMBEC 2017, IFMBE Proceedings}, 65, 2017 
\end{thebibliography}


\end{document}
