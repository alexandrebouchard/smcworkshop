\documentclass[12pt]{article}

\usepackage{amsmath}
\usepackage{url}
\newcommand{\postertitle}[1]{{\Large\bf #1}\\[12pt]}
\newcommand{\authors}[1]{\emph{#1}\\}
\newcommand{\affiliations}[1]{{#1}\\}
\newcommand{\contacts}[1]{{#1}}

%%%%%%%%%%%%%%%%%%%%%%%%%%%%%%%%%%%%%%%%%%%%%%%%%%%%%%%%%%%%%%%%%%%%%%
\begin{document}

\begin{center}
\vspace*{0.5cm}
%
\postertitle{Bayesian modelling and computation for surface-enhanced Raman spectroscopy}
%
\authors{Matthew Moores$^\star$, Kirsten Gracie, Jake Carson, Karen Faulds, Duncan Graham, and Mark Girolami} % please mark the name of the person(s) presenting the poster with a star
% 
\affiliations{Department of Statistics, University of Warwick}
%
\contacts{\url{http://warwick.ac.uk/mmoores}} % URL or email address of contact person
%
\vspace*{0.3cm}
\end{center}

%%%%%%%%%%   Type your abstract below
Raman spectroscopy can be used to identify molecules by the characteristic scattering of light from a laser. Each Raman-active dye label has a unique spectral signature, comprised by the locations and amplitudes of the peaks. The presence of a large, nonuniform background presents a major challenge to analysis of these spectra. We introduce a sequential Monte Carlo (SMC) algorithm to separate the observed spectrum into a series of peaks plus a smoothly-varying baseline, corrupted by additive white noise. The peaks are modelled as Lorentzian, Gaussian or Voigt functions, while the baseline is estimated using a penalised cubic spline. Our model-based approach accounts for differences in resolution and experimental conditions. We incorporate prior information to improve identifiability and regularise the solution. By utilising this representation in a Bayesian functional regression, we can quantify the relationship between molecular concentration and peak intensity, resulting in an improved estimate of the limit of detection. The posterior distribution can be incrementally updated as more data becomes available, resulting in a scalable algorithm that is robust to local maxima. These methods have been implemented as an R package, using RcppEigen and OpenMP.

%%%%%%%%%%%   References
%If you have references, you can produce a .bbl file using bibtex
%and copy/paste thebibliography here
\begin{thebibliography}{1}	
	\bibitem{label}
	M. T. Moores, K. Gracie, J. Carson, K. Faulds, D. Graham \& M. Girolami (2016).
	\newblock Bayesian modelling and quantification of Raman spectroscopy.
	\newblock {\em arXiv preprint}, \texttt{arXiv:1604.07299} [stat.AP]
\end{thebibliography}


\end{document}
