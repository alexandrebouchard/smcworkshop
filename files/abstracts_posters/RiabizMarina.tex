\documentclass[12pt]{article}

\usepackage{amsmath}
\usepackage{url}
\newcommand{\postertitle}[1]{{\Large\bf #1}\\[12pt]}
\newcommand{\authors}[1]{\emph{#1}\\}
\newcommand{\affiliations}[1]{{#1}\\}
\newcommand{\contacts}[1]{{#1}}

%%%%%%%%%%%%%%%%%%%%%%%%%%%%%%%%%%%%%%%%%%%%%%%%%%%%%%%%%%%%%%%%%%%%%%
\begin{document}

\begin{center}
\vspace*{0.5cm}
%
\postertitle{A central limit theorem with application to inference in $\alpha$-stable regression models}
%
\authors{Marina Riabiz$^\star$ and Tohid Ardeshiri and Simon J. Godsill} % please mark the name of the person(s) presenting the poster with a star
% 
\affiliations{Department of Engineering, University of Cambridge}
%
\contacts{\url{mr622@cam.ac.uk}} % URL or email address of contact person
%
\vspace*{0.3cm}
\end{center}

%%%%%%%%%%   Type your abstract below
It is well known that the $\alpha$-stable distribution, while having no closed form density function
in the general case, admits a Poisson series representation (PSR) in which the terms of the
series are a function of the arrival times of a unit rate Poisson process. In our previous
work we have shown how to carry out inference for regression models using this series representation,
which leads to a very convenient conditionally Gaussian framework, amenable
to straightforward Gaussian inference procedures. The PSR has to be truncated to a finite
number of terms for practical purposes. The residual terms have been approximated in
our previous work by a Gaussian distribution with fully characterised moments. In this
paper we present a new Central Limit Theorem (CLT) for the residual terms which serves
to justify our previous approximation of the residual as Gaussian. Furthermore, we provide
an analysis of the asymptotic convergence rate expressed in the CLT.


%%%%%%%%%%%   References
%If you have references, you can produce a .bbl file using bibtex
%and copy/paste thebibliography here
%\begin{thebibliography}{1}	
%	\bibitem{label}
%	A.~Anderson 
%	\newblock Novel theory for methods
%	\newblock {\em Journal of Theory and Methods}, 1(2):1--12, 2017.	
%\end{thebibliography}


\end{document}
