\documentclass[12pt]{article}

\usepackage{amsmath}
\usepackage{url}
\newcommand{\postertitle}[1]{{\Large\bf #1}\\[12pt]}
\newcommand{\authors}[1]{\emph{#1}\\}
\newcommand{\affiliations}[1]{{#1}\\}
\newcommand{\contacts}[1]{{#1}}

%%%%%%%%%%%%%%%%%%%%%%%%%%%%%%%%%%%%%%%%%%%%%%%%%%%%%%%%%%%%%%%%%%%%%%
\begin{document}

\begin{center}
\vspace*{0.5cm}
%
\postertitle{Phenotypic modelling of the polyfunctional T cell response}
%
\authors{Nicola Trendel} 
%
\vspace*{0.3cm}
\end{center}

T cells are integral parts of the adaptive immune system. Binding of the T cell receptor (TCR) to antigenic ligands can result in multiple different functional T cell responses but how these responses depend on the antigen affinity and dose is incompletely understood. Here, we derive a phenomenological model that is consistent with experimental phenotypes. We systematically measured TCR levels and cytokine release in response to antigens of varying affinity and dose at different time points after stimulation. Ligand-induced TCR internalisation is fast and irreversible on the experimental time scale. Consequently, cells stimulated with high doses of antigen shut down cytokine secretion earlier than cells stimulated with lower doses of antigen. A receptor internalisation model with readout-specific activation thresholds explains experimentally observed features. Next, we aim to systematically search models of equal or lesser complexity to identify a minimal, coarse-grained model encompassing only those assumptions necessary to explain the data, thus avoiding complexity bias. Our approach will likely include a combination of automated adaptive inference1 and ABC-SMC.



\end{document}
