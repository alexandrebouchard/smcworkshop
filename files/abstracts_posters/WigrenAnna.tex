\documentclass[12pt]{article}

\usepackage{amsmath}
\usepackage{url}
\newcommand{\postertitle}[1]{{\Large\bf #1}\\[12pt]}
\newcommand{\authors}[1]{\emph{#1}\\}
\newcommand{\affiliations}[1]{{#1}\\}
\newcommand{\contacts}[1]{{#1}}

%%%%%%%%%%%%%%%%%%%%%%%%%%%%%%%%%%%%%%%%%%%%%%%%%%%%%%%%%%%%%%%%%%%%%%
\begin{document}

\begin{center}
\vspace*{0.5cm}
%
\postertitle{Bias-variance trade-off for high-dimensional particle filters using artificial process noise}
%
\authors{Anna Wigren$^\star$, Fredrik Lindsten and Lawrence Murray } % please mark the name of the person(s) presenting the poster with a star
% 
\affiliations{Department of Information Technology, Uppsala University}
%
\contacts{\url{anna.wigren@it.uu.se}} % URL or email address of contact person
%
\vspace*{0.3cm}
\end{center}

%%%%%%%%%%   Type your abstract below
For high-dimensional systems the particle filter suffers from degenerated particle weights both when using the standard and the optimal proposal density. However, in many practical applications the optimal proposal can significantly outperform the standard proposal and provide competitive results for fairly high-dimensional systems. Furthermore, the benefit of using the optimal proposal has been shown to increase when the magnitude of the process noise increases \cite{label}. The optimal proposal is, however, difficult to use in practice since it is only possible to sample from it for a few specific types of models, such as the linear Gaussian model, but the system model can be approximated to enable the use of the optimal proposal. Here the aim is to approximate by first propagating according to the standard proposal and then adding artificial process noise. This introduces a bias but will also reduce the variance since it is possible to use the optimal proposal for the additional artificial noise step. The magnitude of the variance of the artificial noise can be varied to find the trade-off between bias and variance which gives the best overall performance. Simulation results show that a clear improvement in performance is possible. 


%%%%%%%%%%%   References
%If you have references, you can produce a .bbl file using bibtex
%and copy/paste thebibliography here
\begin{thebibliography}{1}	
	\bibitem{label}
	Chris Snyder, Thomas Bengtsson, and Mathias Morzfeld.
	\newblock Performance bounds for particle filters using the optimal proposal.
	\newblock {\em Monthly Weather Review}, 143(11):4750--4761, 2015.	
\end{thebibliography}


\end{document}
