\documentclass[12pt]{article}

\usepackage{amsmath}
\usepackage{url}
\newcommand{\postertitle}[1]{{\Large\bf #1}\\[12pt]}
\newcommand{\authors}[1]{\emph{#1}\\}
\newcommand{\affiliations}[1]{{#1}\\}
\newcommand{\contacts}[1]{{#1}}

%%%%%%%%%%%%%%%%%%%%%%%%%%%%%%%%%%%%%%%%%%%%%%%%%%%%%%%%%%%%%%%%%%%%%%
\begin{document}

\begin{center}
\vspace*{0.5cm}
%
\postertitle{Maximum likelihood estimation by re-using the particles}
%
\authors{Andreas Svensson$^\star$, Fredrik Lindsten and Thomas B. Sch\"on} % please mark the name of the person(s) presenting the poster with a star
% 
\affiliations{Department of Information Technology, Uppsala University}
%
\contacts{\url{http://www.it.uu.se/katalog/andsv164}} % URL or email address of contact person
%
\vspace*{0.3cm}
\end{center}

%%%%%%%%%%   Type your abstract below
In the interest of maximum likelihood parameter estimation in nonlinear state-space models, we make the following observations concerning the particle filter:
\begin{itemize}
	\item The propagation of the states from time t-1 to time t does not have to depend on the model parameters, but can be done using an arbitrary proposal.
	\item The resampling does not have to be done with respect to the importance weights, but can be made with respect to any weights.
\end{itemize}
These two points suggest that we first can run the particle filter with an initial parameter value to estimate the likelihood, and later also compute a likelihood estimate for another parameter value based on the very same particles. We demonstrate in simulated examples that this approach can provide competitive results in certain cases. A similar idea was proposed by \cite{label}, but has, to the best of our knowledge, not previously been fully explored for the use of parameter estimation.


%%%%%%%%%%%   References
%If you have references, you can produce a .bbl file using bibtex
%and copy/paste thebibliography here
\begin{thebibliography}{1}	
	\bibitem{label}
	Fran\c{c}ois~Le~Gland.
	\newblock Combined use of importance weights and resampling weights in sequential Monte Carlo methods
	\newblock {\em ESAIM: Proc.}, 19:85--100, 2007.	
\end{thebibliography}


\end{document}
